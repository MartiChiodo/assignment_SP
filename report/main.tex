\documentclass{article}

% Language setting
\usepackage[english]{babel}

% Set page size and margins
% Replace `letterpaper' with `a4paper' for UK/EU standard size
\usepackage[a4paper,top=2cm,bottom=2cm,left=2cm,right=2cm,marginparwidth=1.75cm]{geometry}

\usepackage[utf8]{inputenc}
\usepackage[T1]{fontenc}
\usepackage{lmodern}

% Useful packages
\usepackage{matlab-prettifier} % per il codice Matlab colorato carino
\usepackage{csvsimple-l3} % per le tabelle
\usepackage{booktabs} % per le tabelle
\usepackage{longtable} % per le tabelle
\usepackage{amsmath}
\usepackage{amsfonts}
\usepackage[thinc]{esdiff}
\usepackage{minted}
\usepackage{bbm}
\usepackage{bm}
\usepackage{optidef}
\usepackage{graphicx}
\usepackage[colorlinks=true, allcolors=blue]{hyperref}
\usepackage{subcaption}   % Per immagini affiancate con didascalie


\title{Report SO}
\author{Chiodo Martina - 343310 \\ Vigè Sophie - 339268}
\date{}

\begin{document}
\begin{titlepage}
    \newgeometry{left=1cm, right=1cm, top=3cm, bottom=3.5cm}  % Margini specifici per questa pagina
    
    \begin{center}
    \includegraphics[width=0.35\textwidth]{img/logo_polito_2021.jpg}\\[1cm] 
    {\huge POLITECNICO DI TORINO}\\[1.5cm]
    \textbf{Corso di Laurea\\in Ingegneria Matematica}\\[3cm]
    
     
    {\huge Report Stochastic Optimization}\\[1cm]
    {\huge GRASP}\\[1cm]
    {\large Chiodo Martina - 343310 \\
    Vigè Sophie - 339268}
    
    
    
    \vfill
    
    Anno Accademico 2024-2025
    \end{center}
    
    \restoregeometry
    
    \end{titlepage}

\section{Data structures}
The problem considered is the scheduling of surgeries and patients' admission in a single healthcare 
system. We are provided with different entities populating the system, and for each of them we created a corresponding class:
\begin{itemize}
    \item Hospital: even if there is only one hospital considered, we use this object to store and update information about 
    the rooms (their capacity, availability at a certain time, sex of the people in it) and the operating theaters (their availability). 
    Here there is also a function (\texttt{creating\_matrix\_dayxroomxpatients}) that creates a 3-dimensional matrix that will be useful to verify constraints: 
    it is a list of length the number of days containing, for each day, a list of length the total number of rooms that contains a list of all the people
    staying in the said room during the said day.
    \item Nurse: nurses are identified by their id and the shifts they are working are given and fixed.
    \item Surgeon: they are treated similarly to the nurses, but their maximum surgery time on available days must be respected (it is an hard 
    constraint, differently from the nurses' one).
    \item Occupant: patients that are already in the hospital at the beginning of the considered period of time.
    \item Patient: patients still to be admitted. Their admission can either be mandatory or not.
    \item Problem: we store here all the dictionaries containing the istances of the previous classes. In this way we can verify 
    the constraints and generate feasible states.
    \item State: a certain schedule for the hospital; some structures are used to memorize it. 
    First, the scheduling of patients' admissions. It is stored in a dictionary which has as key the id of the patient and 
    as a value a vector of three elements: the operating theatre his surgery will take place in, the date of the surgery (which 
    coincides with the admission date and it is setted to -1 if the non mandatory patient is not admitted within the considered 
    scheduling period) and the room the patient is staying in.
    Then, a dictionary that has as key nurses' id and as value a vector (with length the total number of shifts in the 
    period of time) containing -1 if the nurse is not working and otherwise the id of the rooms he is assigned to (they can be even more than one per shift). 
    Lastly the scheduling for the operating theaters: a dictionary with key the id of the OT and as value a vector (with length the total number 
    of days) that for each day contains the list of ids of the patients that undergo surgery in that OT that day.
    We store here also the 3-dimensional matrix introduced above, when created using the method of the Hospital class.
\end{itemize}


\section{Hard and soft constraints}
Hard constraints must be satisfied for a state to be feasible, while soft constraints if not satisfied produce a penalty in the 
objective function. The costs of these penalties are given and fixed.

In order to verify the feasibility of a given state we start by checking that all the operating theaters, rooms and admission dates exist.

Then, for each nurse we check that they have at least a room assigned for the shifts they are supposed to be working and no room assigned during 
the shifts they are unavailable.

For every room, we check that it is not incompatible for the people staying in it, that there is not gender mix among the patients and 
that the maximum capacity of the room is respected. We also need to check that there is always at least a nurse assigned to it (if the room is not empty).

For every mandatory patient we check that they are admitted within the scheduling period and, more specifically, their admission date
should be between their release date and their due date. For non-mandatory patients that are admitted, the check is just on the release date,
since they do not have a due date.

Lastly, for every surgeon and operating theatre, we check that the total amount of time of the scheduled
surgeries does not exceed the maximum available for every considered day.

For soft constraints, we penalize in the eveluation of the objective function the following: admission delays (we want to minimize the number of 
days between a patient's release date and their admission) and unscheduled non-mandatory patients, opening of operating theaters, surgeon transfers. 
For nurses, the eventual excess of their maximum workload, the continuity of care (we try to minimize the number of different nurses that take care of 
a patient) and the minimum skill required for a patient, that might not be respected by the nurse.

\section{Neighborhoods generation}

\section{Outcome}

\end{document}