\documentclass{article}

% Language setting
\usepackage[english]{babel}

% Set page size and margins
% Replace `letterpaper' with `a4paper' for UK/EU standard size
\usepackage[a4paper,top=2cm,bottom=2cm,left=2cm,right=2cm,marginparwidth=1.75cm]{geometry}

\usepackage[utf8]{inputenc}
\usepackage[T1]{fontenc}
\usepackage{lmodern}

% Useful packages
\usepackage{matlab-prettifier} % per il codice Matlab colorato carino
\usepackage{csvsimple-l3} % per le tabelle
\usepackage{booktabs} % per le tabelle
\usepackage{longtable} % per le tabelle
\usepackage{amsmath}
\usepackage{amsfonts}
\usepackage[thinc]{esdiff}
\usepackage{minted}
\usepackage{bbm}
\usepackage{bm}
\usepackage{optidef}
\usepackage{graphicx}
\usepackage[colorlinks=true, allcolors=blue]{hyperref}
\usepackage{subcaption}   % Per immagini affiancate con didascalie


\title{Report SO}
\author{Chiodo Martina - 343310 \\ Vigè Sophie - 339268}
\date{}

\begin{document}
\begin{titlepage}
    \newgeometry{left=1cm, right=1cm, top=3cm, bottom=3.5cm}  % Margini specifici per questa pagina
    
    \begin{center}
    \includegraphics[width=0.35\textwidth]{img/logo_polito_2021.jpg}\\[1cm] 
    {\huge POLITECNICO DI TORINO}\\[1.5cm]
    \textbf{Corso di Laurea\\in Ingegneria Matematica}\\[3cm]
    
     
    {\huge Report Stochastic Optimization}\\[1cm]
    {\huge GRASP}\\[1cm]
    {\large Chiodo Martina - 343310 \\
    Vigè Sophie - 339268}
    
    
    
    \vfill
    
    Anno Accademico 2024-2025
    \end{center}
    
    \restoregeometry
    
    \end{titlepage}

\section{Data structures}
The problem considered is the scheduling of surgeries and patients' admission in a single healthcare 
system. We are provided with different entities populating the system, and for each of them we created a corresponding class:
\begin{itemize}
    \item Hospital: even if there is only one hospital considered, we use this object to store and update information about 
    the rooms (their capacity, availability at a certain time, sex of the people in it) and the operating theaters (their availability). 
    Here there is also a function (\texttt{creating\_matrix\_dayxroomxpatients}) that creates a 3-dimensional matrix that will be useful to verify constraints: 
    it is a list of length the number of days containing, for each day, a list of length the total number of rooms that contains a list of all the people
    staying in the said room during the said day.
    \item Nurse: nurses are identified by their id and the shifts they are working are given and fixed.
    \item Surgeon: they are treated similarly to the nurses, but their maximum surgery time on available days must be respected (it is an hard 
    constraint, differently from the nurses' one).
    \item Occupant: patients that are already in the hospital at the beginning of the considered period of time.
    \item Patient: patients still to be admitted. Their admission can either be mandatory or not.
    \item Problem: we store here all the dictionaries containing the istances of the previous classes. In this way we can verify 
    the constraints and generate feasible states.
    \item State: a certain schedule for the hospital; some structures are used to memorize it. 
    First, the scheduling of patients' admissions. It is stored in a dictionary (\texttt{dict\_admission}) which has as key the id of the patient and 
    as a value a vector of three elements: the operating theatre his surgery will take place in, the date of the surgery (which 
    coincides with the admission date and it is setted to -1 if the non mandatory patient is not admitted within the considered 
    scheduling period) and the room the patient is staying in.
    Then, a dictionary (\texttt{nurses\_shifts}) that has as key nurses' id and as value a vector (with length the total number of shifts in the 
    period of time) containing -1 if the nurse is not working and otherwise the id of the rooms he is assigned to (they can be even more than one per shift). 
    Lastly the scheduling for the operating theaters: a dictionary (\texttt{scheduling\_OTs}) with key the id of the OT and as value a vector (with length the total number 
    of days) that for each day contains the list of ids of the patients that undergo surgery in that OT that day.
    We store here also the 3-dimensional matrix introduced above, when created using the method of the Hospital class.
\end{itemize}


\section{Hard and soft constraints}
Hard constraints must be satisfied for a state to be feasible, while soft constraints if not satisfied produce a penalty in the 
objective function. The costs of these penalties are given and fixed.

In order to verify the feasibility of a given state we start by checking that all the operating theaters, rooms and admission dates exist.

Then, for each nurse we check that they have at least a room assigned for the shifts they are supposed to be working and no room assigned during 
the shifts they are unavailable.

For every room, we check that it is not incompatible for the people staying in it, that there is not gender mix among the patients and 
that the maximum capacity of the room is respected. We also need to check that there is always at least a nurse assigned to it (if the room is not empty).

For every mandatory patient we check that they are admitted within the scheduling period and, more specifically, their admission date
should be between their release date and their due date. For non-mandatory patients that are admitted, the check is just on the release date,
since they do not have a due date.

Lastly, for every surgeon and operating theatre, we check that the total amount of time of the scheduled
surgeries does not exceed the maximum available for every considered day.

For soft constraints, we penalize in the eveluation of the objective function the following: admission delays (we want to minimize the number of 
days between a patient's release date and their admission) and unscheduled non-mandatory patients, opening of operating theaters, surgeon transfers. 
For nurses, the eventual excess of their maximum workload, the continuity of care (we try to minimize the number of different nurses that take care of 
a patient) and the minimum skill required for a patient, that might not be respected by the nurse.

\section{Neighborhoods generation}
The \textit{Greedy Randomized Adaptive Search Procedure} (GRASP) is based on repeated local search in the neighborhood 
of a given point. We generate, for each given state, a set of states obtained by perturbation of the given one. When doing 
so, we try to check some of the hard constraints in order not to generate too many neighbors that are not feasible states.
The perturbations performed are the following:
\begin{itemize}
    \item If a non-mandatory patient was not admitted within the scheduling period, we admit him. We generate several new states, 
    admitting the considered patient in every possible date after his release date and in every possible room.
    \item If a patient is mandatory, we generate new states by adding or subtracting a day to their admission date (if this makes sense, for example 
    we cannot add +1 to the admission date of a patient that was admitted on the last day of the scheduling period).
    \item If a non-mandatory patient was admitted, we generate a new state where we do not admit him within the scheduling period.
    \item For every admitted patient, we perturbate the room they are assigned to, by adding or subtracting 1 to the id of the room (or just adding or 
    subtracting if they are in the first or last room).
    \item For every admitted patient, we change the operating theater they are assigned to by choosing randomly between the available ones.
    \item For all the nurses, we generate some new states where we exchange the rooms that two nurses working in the same shift are asssigned to.
\end{itemize}


\section{GRASP implementation and outcome}
The idea of GRASP is the following: given an initial feasible state, we generate and explore all its neighborhood,
updating the current solution with the best found in it. If such a current solution is found and it is different from the previous one, 
we repeat the local search step in a neighborhood of this new state. Otherwise we generate a new feasible starting state and repeat.

In order to generate randomly a feasible state we have to check that all the hard constraints are respected. If a state is generated 
completely randomly it will be very likely unfeasible. So, in the class \texttt{problem}, the function \texttt{generating\_feasible\_state} 
is built in such a way that it returns a feasible state that is partially random, but generated taking into account some of the hard constraints 
in order not to discard too many generated states and save computational time.

The admission date for each mandatory patient is chosen randomly between their release date to their due date, exluding the days their assigned surgeon 
is not working. The room a patient is assigned to is chosen randomly between the ones that are not already full and we consider every room to be "labeled" with the sex 
of the patients in it, so we exclude those belonging to the opposite sex. For the sake of simplicity we don't admit non-mandatory patients (this is not restrictive, since 
in the generation of neighborhood we will admit them). We assign the nurses to the rooms by counting how many nurses are working in each shifts; in this way we decide 
how many rooms should a nurse cover (for example if we have 9 working nurses and 9 rooms we assign one room per nurse, while if we only have 4 nurses, we give 2 rooms each to 
the first three and 3 to the last nurse). The room each nurse is assigned to is chosen randomly.

All the elements needed to run the GRASP on the given problem have been built. The result of a run 
is the following.

\begin{figure}[H]
    \centering
    % Prima immagine
    \begin{subfigure}[t]{0.45\textwidth}  % Larghezza del 45% del testo
        \centering
        \includegraphics[width=\textwidth]{img/1_restart.png}
        \caption{One restart allowed}
    \end{subfigure}
    \hspace{1cm} %spaziatura tra le immagini
    % Seconda immagine
    \begin{subfigure}[t]{0.45\textwidth}
        \centering
        \includegraphics[width=\textwidth]{img/3_restarts.png}
        \caption{Three restarts allowed}
    \end{subfigure}
    % Didascalia generale
    \caption{Cost of generated feasible solutions. The red line corresponds to restarts occurring when no improving solution is found in the neighborhood of the current point.}
    \label{GRASP}
\end{figure}

In the first case, the best value of the objective function found is 11466, while in the second case it is 11127.

Red lines indicates when the method restarts from scratch, picking randomly a new feasible state. That's why the 
value of the objective function can increase here, while it is always non-increasing in the other points, where exploration of the 
neighborhoods is performed.

\section{Results}
We have run few experiments using the datasets made available by the Integrated Healthcare Timetabling Competition. 
These have been run by fixing the number of restart equal to $2$.

The datasets are very different from one another, also differing in the length of the timeline, consequently it is normal that the objective function assume values of different magnitude.

\begin{table}[h!]
    \centering
    \begin{tabular}{|c|c|c|c|c|}
        \hline
        \textbf{Test 1} & \textbf{Test 2} & \textbf{Test 3} & \textbf{Test 4} & \textbf{Test 5}  \\ \hline
        5343 & 7417 & 11127 & 6382 & 16885   \\ \hline
        3177 & 1583 & 10184 & 2332 & 15713  \\ \hline
    \end{tabular}
    \caption{Comparison between the solution found by the GRASP solver (first row) and the actual minimum value of the objective function (second row).}
\end{table}

Due to the fact that the GRASP solver belongs to the class of local optimizers, there is no guarantee of convergence to the global minimum: the algorithm can easily get stuck in local minima. Therefore, it is not surprising that the solutions found by our solver are always greater than the actual minimum value of the objective function.

In addition, the performance of the GRASP solver is largely affected by the choice of the initial guess. In our code, the initial guess is generated without the aim of improving the objective function, and thus it is possible that the algorithm starts from points very far from the minimum, making it more difficult to reach it.

\end{document} 